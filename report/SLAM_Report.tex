\documentclass[10pt]{IEEEtran}
\usepackage{lipsum}
\begin{document}
\title{SLAM paper draft}
\author{Daniel~Snyder
\thanks{Put thanks here}
}
\maketitle
\begin{abstract}
One of the fundamental problems in autonomous mobile robotics is that of simultaneous
localization and mapping (SLAM).  Solving SLAM involves the robot simultaneously
determining both its position in the environment (localization) and what its environment
is (mapping), which are problems dependent on each other.  This problem is essential to
many of the other problems present in autonomous mobile robotics.  If a robot does not
know its location and its environment it cannot complete tasks such as navigation.  
 
There exists many approaches to solving the SLAM problem, and each approach has
its own drawbacks and advantages.  The goal of this project is to implement and
compare multiple solution methods to determine their advantages within a specific
domain.  The solutions will be evaluated in the context of an aquatic autonomous surface
vehicle (boat) designed and built by the UM::Autonomy student team at the University
of Michigan for participation in the International Roboboat competition hosted by
the Association for Unmanned Vehicle Systems International (AUVSI).   
  
In particular a graph-based solution using the incremental smoothing and mapping
(ISAM) method, and a solution using an occupancy grid for mapping along with a 
particle filter for localization (GM-MCL) will be implemented and compared
to the Extended Kalmann Filter (EKF) based SLAM that has been implemented by previous
UM::Autonomy team members.  Their performance will be evaluated based not only on
actual task performance, but also on the ease of understanding and maintainability
of the system over time.  The focus on ease of understanding and maintainability
arises from past challenges the current EKF SLAM implementation has presented to the team. 
\end{abstract}

\section{Overview of the SLAM Problem}
SLAM is a problem fundamental to autonomous robotics.  In order for a robot to accurately
traverse and interact with its environment, it needs to know what the environment is 
(mapping) and where it is in the environment (localization).  While for some applications it
is possible to generate a map in advance, for many applications, it is necessary to perform 
both localization and mapping at the same time.  The difficulty lies in that in order to 
accurately map, the robot must know its location, and that in order to find its location, the
robot must have some form of a map.  SLAM techniques attempt to solve this problem though 
various methods.  This paper will look at two such methods.

\section{Research Goals}
The primary goal of this reasearch is to create a SLAM system that 
UM::Autonomy can use for years to come.  Inherent within this goal are a few criteria that
are required for the goal to be achieved. These criteria fall into three categories:
performance, useability, and maintainability.

\subsection{Performance}
The first set of criteria that this research must satisfy is that of performance
requirements.  The end product must work reasonably well in a multitude of environment and
must fit within the computing capabilities of the surface vehicle.  The end product
should be able to perform online SLAM, approximating a solution to the SLAM problem in 
real time, while allowing the team to run other code simultaneously.

\subsection{Useability}
Beyond performance, the SLAM system produced must be easy to use.  A majority of the 
people using the system will not have much knowledge of how the SLAM problem is solved,
so its use cannot require multiple steps or indepth knowledge of the solution method.

\subsection{Maintainability}
The maintainability of the system is the most crucial to assuring its prolonged useage.
In the past the team has neglected the SLAM system due to its complexity and over time,
this has caused the system to degrade in quality.  Large consideration therefore will
be put into code quality and design.  The code should be designed with maintainability in
mind to prevent degredation over time.

\section{Document Purpose}
This document is meant to serve two purposes.  The first purpose is to provide a 
technical explaination of the research conducted on the SLAM problem.  The second
purpose is to serve as a technical reference for UM::Autonomy for understanding and extending
the SLAM solution that is presented.

\section{Capabilities of the Autonomous Surface Vehicle}
As the system was designed for use with the Autonomouse Surface Vehicle (ASV) that is
built by UM::Autonomy, the solution approaches are heavily dependant on the capabilities of
the platform.  The ASV is equipped with a Global Positioning System (GPS), a fiber optic 
gyroscope (FOG) and a panning Lidar with a 270 degree field of view.  Together, these sensors
allow the ASV to obtain an approximate global pose and to survey the local area.

Beyond the current sensors on the ASV, the team is also planning to add an inertial 
measurement unit (IMU) and a compass.  In order to meet maintainability requirements, it
is important that these sensors are also considered when performing the research.

\section{Solution approaches}

\subsection{Incremental Smoothing and Mapping}
ISAM draws on the work of Kaess et al.\cite{}
\cite{}

\subsection{Occupancy Grid based Mapping and Monte-Carlo Localization}
In contrast to ISAM, the GM-MCL is not a feature based SLAM method.  Rather than representing
the map and the robot's location in it as a set of state vectors, the map is represented
as a grid, where each cell contains a probability that the area of the world it represents
contains some object.  

Once the occupancy grid has been created, Monte-Carlo Localization (MCL) is used to find the
location of the robot within the map.  Proposed by Dellaert et al \cite{MCL}

\section{SLAM Related Utilities}
In order to create a SLAM program that is performant, useable, and maintainable, a few 
SLAM related utilities were created during development.  These utilities are a 3D Point Cloud
that is generated from a panning LIDAR, and a simulated compass.

\subsection{3D Point Cloud Generation}
% Make sure to note about the adding of hits
One feature of the robotic boat is a panning LIDAR that allows for generation of a three 
dimensional point cloud.  This point cloud allows for detection of objects floating on the
water's surface in addition to objects that are raised off the water's surface.  

% put technical details in here
\subsection{Compass spoofing}
A challenge that the team has faced in the past few years is the lack of a compass.  
The boat uses the NED (North East Down) Coordinate System, so if North cannot be found, 
the coordinate system may be set incorrectly.  In the past the team has solved this by
starting the boat facing North, a process that is subject to human error. %continue here

\section{In depth explaination of occupancy grid and particle filter approach}

\subsection{Initialization}% may not be needed

\subsection{Filling the Map}

\subsection{Creating Particles to be Filtered}
In prior implementations particles for the current generation were created from sampling from
the previous generation of particles, and applying a change in location based on odometry 
data plus some error.  For the ASV however, there is currently no sense of odometry that 
can be used to link the previous generation to the current generation.  Additionally, the 
team will be adding an IMU to the ASV, giving a sense of odometry.  To allow for easy 
integration of the IMU, two types of particles, distinguished by the that creates them,
were developed; generational particles and non-generational particles.

%clean up this paragraph
After generating the set of particles from the two methods, the particles only contain X and
Y coordinates.  To get the theta for each particle, each particle samples from a gaussian
distribution centered at the last FOG measurement.  This generates a set of possible yaws
for the ASV.  These particles are then weighted based on lidar data.

\subsubsection{Generational Particles}
Generational particles are particles that are created by sampling the previous generation of
particles, and predicting them forward based on odometry data. Currently this is done using
the change in GPS data to predict a particle forward.  However, since GPS gives a sense of 
global location rather than change in local location, it is not an ideal source of odometry
data.  It is recommended that an IMU is used to generate the odometry data, even though that
is outside of the ASV's current capabilities.  Despite the lack of an IMU, this method was
developed to ease extension of the code.  Having the framework in place to generate 
generational particles allows for easy IMU integration, preventing a potential source of 
code quality degredation.

\subsubsection{Non-Generational Particles}
Non-generational particles are particles created from sampling a two dimensional gaussian
distribution whose mean is the last received GPS coordinate.  This creates a set of particles
that are independant of the previous generation.  The benefits of this approach lie in its 
ability to correct for symmetric environments.  Since GPS is a global position, it is 
more robust to errors created by a misprediction in locally symmetric maps.  
By using non-generational particles, the progam assures that there always exists some 
particles in the area closest to the global location of the ASV. %reword this

%This may not be the correct place for this
Another benefit to the non-generational approach is that it allows for GPS only localization.
By using only non-generational particles and weighting based only on GPS, the pose will 
be determined based only on the GPS location.

\subsection{Weighting the particles}
\subsubsection{Point Cloud Weight Generation}
\subsubsection{GPS Based Weight Generation}
\subsubsection{Combining Various Sensors}


\section{Observations}
\subsection{ability to recover from symmetric environments}

\section{Results}
\subsection{Analysis of Maintainability of Two Approaches}
\subsection{Overall Results}
\subsection{Effects of varying grid size}
\subsection{Effects of varying number of particles}
\subsection{Effects of varying GPS Gaussian Distribution Sigma}
\subsection{Effects of increasing number of generational particles}
\subsection{Effects of changing weights for combining sensors}
\subsection{Effects of changing lidar hit and miss likelihoods}


\section{Conclusions}

\section{Recommended Approach to Future Extensions}
\subsection{Adding a compass}
\subsection{Adding an IMU}

\appendix
\section{Tuneable Constants}

\bibliographystyle{IEEEtran}
\bibliography{IEEEabrv,SLAM_Report}
\end{document}
